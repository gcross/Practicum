%@+leo-ver=4-thin
%@+node:gcross.20090803112359.1303:@thin TermResumming.tex
%@@language latex
\documentclass[onecolumn,amsmath,amssymb,nofootinbib,floatfix]{revtex4}

\input{notes-prelude.in}

\title{Term Resumming}

\begin{document}

Suppose that we want to compute a sum of the form
$$\sum_{K\subseteq \{1,\dots,N\}\atop |K|=m} \prod_{k\in K} f(x_k).$$
At first glance, this might seem to need exponential time to compute.  However, we shall show that this sum can be reduced to a series of matrix multiplications.

First observe that since $K$ is a set, there can be no duplicate values of $k$, and hence we can think of this sum as being over all monotonically strictly increasing sequences of integers ranging between $1$ and $N$;  put another way, this sum of products can be rewritten in the nested sum form
$$\sum_{1 \le k_1 \le N} \sum_{k_1 < k_2 \le N} \dots \sum_{k_{m-1} < k_m \le N} \prod_{i=1}^m f(x_{k_i}).$$

Note that the summation signs do not commute since the range of each summand is dependent on the previous summands.  To get around this, we define the following matrix
$$\gamma_{ij}:=
\begin{cases}
1 & i < j \\
0 & \text{othewise}
\end{cases}
$$

With this, we can now write the expression above as
$$
\sum_{
1\le k_1\le N \atop {
1\le k_2\le N \atop {
\vdots \atop {
1\le k_m\le N
}}}}
\gamma_{k_1k_2}\gamma_{k_2k_3}\dots\gamma_{k_{m-1}m}
\prod_{i=1}^m f(x_{k_i})
$$
Now define the vectors
$$L_i := f(x_i), \quad R_i = 1,$$
and the matrix
$$M_{ij} = \gamma_{ij}f(x_j),$$
and we see that the expression above can be written as
$$
\sum_{
1\le k_1\le N \atop {
1\le k_2\le N \atop {
\vdots \atop {
1\le k_m\le N
}}}}
L_{k_1}M_{k_1k_2}M_{k_2k_3}\dots M_{k_{m-1}k_m} R_{k_m} = L\cdot M \cdot M \cdots \cdot R
$$

We could stop now and declare victory, but there is a further optimization that can be performed to reduce the size of this sum.  Observe that no sequence $\{k_i\}_{1\le i\le m}$ can start with an integer greater than $N-m+1$ or else the end of the sequence (being of length $m$) would exceed $N$.  Thus, the values of $k_1$ that result in non-vanishing terms restricted to $1\le k_1\le N-m+1$.  By similar reasoning, we see that $2\le k_2 \le N-m+2$, etc., so that the sum above can equivalently be written as
$$
\sum_{
1\le k_1\le N-m+1 \atop {
2\le k_2\le N-m+2 \atop {
\vdots \atop {
m-1\le k_{m-1}\le N-1 \atop {
m\le k_m\le N
}}}}}
L_{k_1}M_{k_1k_2}M_{k_2k_3}\dots M_{k_{m-1}k_m} R_{k_m}
=\sum_{
1\le k_1\le N-m+1 \atop {
1\le k_2\le N-m+1 \atop {
\vdots \atop {
1\le k_{m-1}\le N-m+1 \atop {
1\le k_m\le N-m+1
}}}}}
L_{k_1}M_{k_1(k_2+1)}M_{(k_2+1)(k_3+2)}\dots M_{(k_{m-1}+m-2)(k_m+m-1)} R_{k_m}
$$

If we define new vectors $\tilde L$ and $\tilde R$ to consist of the first $N-m+1$ elements of respectively $L$ and $R$ and matrices
$$
\tilde M^{l}_{ij} := M_{(i+(l-1))(j+l)}= \gamma_{i(j+1)}f(x_{j+l}),
$$
then we see that we can rewite the sum above in the form
$$
\sum_{
1\le k_1\le N-m+1 \atop {
1\le k_2\le N-m+1 \atop {
\vdots \atop {
1\le k_{m-1}\le N-m+1 \atop {
1\le k_m\le N-m+1
}}}}}
\tilde L_{k_1}\tilde M^{(1)}_{k_1k_2}\tilde M^{(2)}_{k_2k_3}\dots \tilde M^{(m-1)}_{k_{m-1}k_m} \tilde R_{k_m}
=\tilde L \cdot \tilde M^{(1)}\cdot \tilde M^{(2)} \cdots \tilde M^{(m-1)}\cdot \tilde R
$$

As expressed above, this product takes $O(m\cdot (N-m)^2)$ time and $O(N-m)$ space.  While this is a great improvement over the original exponential sum, we can do stil better.  Observe that
$$\paren{\tilde L\cdot \tilde M^{(l)}}_j
= \sum_i L_i M^{(l)}_{ij}
= \sum_i L_i \gamma_{i(j+1)} f(x_{j+l})
= \underbrace{\paren{\sum_{i\le j} L_i }}_{=:S_j} f(x_{j+l})
= S_j f(x_{j+l}),$$
where observe that $S_j$ can equivalently be defined recursively by
$$S_j := S_{j-1}+L_j,\quad S_1 = L_1.$$
Thus we observe that computing $(\tilde L\cdot \tilde M^{(l)})_j$ for each value of $j$ takes $O(1)$ time assuming that we have kept around $S_j$ for the previous value of $j$, and so we conclude that the full vector $\tilde L\cdot \tilde M^{(l)}$ can be computed in $O(N-m)$ time, and hence the full sum can be computed in merely $O\paren{m(N-m)}$ time.

\end{document}
%@-node:gcross.20090803112359.1303:@thin TermResumming.tex
%@-leo
