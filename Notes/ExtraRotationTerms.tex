%@+leo-ver=4-thin
%@+node:gcross.20090730151355.1298:@thin ExtraRotationTerms.tex
%@@language latex
\documentclass[onecolumn,,amsmath,amssymb,nofootinbib,floatfix]{revtex4}

\input{notes-prelude.in}

\begin{document}

Recall that the operator for angular momentum about the $z$-axis is given by,
$$\textbf{L}_z = \hat{\textbf{z}}\cdot \vec{\textbf{L}} = -i\hbar\hat{\textbf{z}}\cdot (\vec{\textbf{r}}\times\vec{\nabla}).$$

We are interested in working with a system in a rotating frame;  we can do this by subtracting a term from the Hamiltonian to obtain
$$H_{rot} = H - \Omega \textbf{L}_z.$$

Since we are working in the fixed phase approximation, we compute the effective Hamiltonian arising from this constraint:
$$\coip{\Psi}{\Omega\textbf{L}_z}{\Psi} = \hbar\Omega\left[\hat{\textbf{z}}\cdot\vec{\textbf{r}}\times(\vec{\nabla}\phi)\right]\psi^2-i\hbar\Omega\psi\left[\hat{\textbf{z}}\cdot\vec{\textbf{r}}\times(\vec{\nabla}\psi)\right]$$

If we can assume that $\phi$ is cylindrically symmetric, then we can make a further simplification.  Recall that the gradient operator expressed in cylindrical coordinates is given by,
$$\pdrv{}{\rho}\hat\rho + \frac{1}{\rho}\pdrv{}{\theta}\hat\theta + \pdrv{}{z}\hat z.$$

If we can assume that all but the $\pdrv{}{\theta}$ derivatives vanish, then we see that
$$
\begin{aligned}
\hat{\textbf{z}}\cdot\vec{\textbf{r}}\times(\vec{\nabla}\phi)
= \hat{\textbf{z}}\cdot\paren{\rho \hat{\rho} + z\hat{\textbf{z}}}\times\paren{\frac{1}{\rho}\pdrv{\phi}{\theta}\hat\theta}
= \pdrv{\phi}{\theta}\\ 
\end{aligned}
$$
and therefore
$$\text{Re}\left[\coip{\Psi}{\Omega \textbf{L}_z}{\Psi}\right] = \hbar \Omega\pdrv{\phi}{\theta}\psi^2.$$

\end{document}
%@-node:gcross.20090730151355.1298:@thin ExtraRotationTerms.tex
%@-leo
