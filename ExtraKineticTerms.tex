\documentclass[twocolumn,showpacs,preprintnumbers,amsmath,amssymb,nofootinbib,pra,floatfix]{revtex4}

\usepackage{mathrsfs,amsthm,clrscode}

\newtheorem{theorem}{Theorem}
\newtheorem{proposition}{Proposition}
\newtheorem{lemma}{Lemma}
\newtheorem{corolary}{Corolary}

\newenvironment{definition}[1][Definition]{\begin{trivlist}
\item[\hskip \labelsep {\bfseries #1}]}{\end{trivlist}}

\newenvironment{example}[1][Example]{\begin{trivlist}
\item[\hskip \labelsep {\bfseries #1}]}{\end{trivlist}}
\newenvironment{remark}[1][Remark]{\begin{trivlist}
\item[\hskip \labelsep {\bfseries #1}]}{\end{trivlist}}

\newcommand{\lst}{\vec}
\newcommand{\set}{\bar}

\newcommand{\Div}{\text{\textbf{div}}\,}

\input{macros.tex}

\begin{document}

\section{Prelude}

Let the wave function be defined by $\Psi$ (implicitly a function of position), where $\Psi=e^{i\phi} \psi$, and $\psi=|\psi|$.  We wish to compute $\coip{\Psi}{\hat H}{\Psi} = \coip{\Psi}{\hat T+\hat V}{\Psi}.$  Since $V$ is diagonal with respect to the basis in which $\Psi$ is expressed, the only ``interesting'' stuff comes from the expectation of $T$, which for a system immersed in a magnetic potential is given by $$\hat T:=(i\nabla + A)^2 = -\Delta + A^2 - 2i(A\cdot \nabla) - i(\Div A).$$

Ergo, we have that
$$
\begin{aligned}
&\coip{\Psi}{\hat T}{\Psi} \\
&\quad = \Psi^* \hat T \Psi, \\
&\quad = \Psi^* \left[-\Delta + A^2 - 2i(A\cdot \nabla) - i(\Div A)\right] \Psi, \\
&\quad = \paren{A+\nabla \phi}^2 + \psi\Delta \psi \\ &\qquad\qquad - i\left\{\Delta \phi + \Div A + 2\psi\nabla\psi \cdot (A+\nabla\phi)\right\}
\end{aligned}
$$

\section{Fixed Phase Approximation}

Consider the fixed phase approximation, in which we assume that $\phi = \theta$ in cylindrical coordinates $(\rho,\theta,z)$.  Then observe that
$$\begin{aligned}
\phi &= \theta,\\
\nabla \phi &= \frac{\hat\theta}{\rho},\\
\Delta \phi &= 0,
\end{aligned}
$$ and so $$\exp{\hat T} = \paren{A+\frac{\hat\theta}{\rho}}^2 + \psi\Delta\psi - i\left\{\Div A + 2\psi\nabla\psi \cdot \paren{A+\frac{\hat\theta}{\rho}}\right\}.$$  


\subsection{No magnetic field}
In the absense of a magnetic field, this reduces to  $$\exp{\hat T} =\rho^{-2} + \psi\Delta\psi - \frac{2i}{\rho}\psi\pdrv{\psi}{\theta}.$$ Since the expectation must be real, it must be that $\psi \pdrv{\psi}{\theta}=0.$  This condition is happily satisfied for wave functions that are rotationally symmetric since then $\pdrv{\psi}{\theta}=0$, but it is not clear how it would be satisfied otherwise.  If there are multiple particles and one is using the Feynman phase approximation, then we must have that $$\psi \paren{\sum_i \frac{1}{\rho_i}\pdrv{\psi}{\theta_i}} = 0.$$

\subsection{Constant magnetic field}

Let $A=h\rho \hat\theta/2$, so that $B=\nabla\times A = h\hat z$.  Then
$$\exp{\hat T} =  \paren{\frac{h}{2}\rho+\rho^{-1}}^2 + \psi\Delta\psi - i\left\{2\psi\pdrv{\psi}{\theta} \paren{\frac{h}{2}\rho + \rho^{-1}}\right\}.$$  

\end{document}
